% Created 2021-08-14 Sa 15:55
% Intended LaTeX compiler: pdflatex
\documentclass[11pt]{article}
\usepackage[utf8]{inputenc}
\usepackage[T1]{fontenc}
\usepackage{graphicx}
\usepackage{grffile}
\usepackage{longtable}
\usepackage{wrapfig}
\usepackage{rotating}
\usepackage[normalem]{ulem}
\usepackage{amsmath}
\usepackage{textcomp}
\usepackage{amssymb}
\usepackage{capt-of}
\usepackage{hyperref}
\author{Marcus Birkenkrahe}
\date{\today}
\title{DSC 101 Fall 2021 Syllabus\\\medskip
\large Lyon College Data Science Program}
\hypersetup{
 pdfauthor={Marcus Birkenkrahe},
 pdftitle={DSC 101 Fall 2021 Syllabus},
 pdfkeywords={},
 pdfsubject={},
 pdfcreator={Emacs 27.2 (Org mode 9.3.7)}, 
 pdflang={English}}
\begin{document}

\maketitle
\section{General Course Information}
\label{sec:org0e9c926}

\begin{itemize}
\item Meeting Times: Tuesday \& Thursday, 13:00-14:15 hrs
\item Meeting place: Derby 209 (CS research lab)
\item Professor: Marcus Birkenkrahe
\item Office: Derby 210
\item Phone: (501) 422-4725
\item Office hours: Mon/Wed/Fri 10:00-10:30 AM; Tue/Thu 4-4:30 PM
\item Text: \emph{The Art of R Programming - A Tour of Statistical Software
Design}, by Norman Matloff, NoStarch Press (2011).
\end{itemize}

\subsection{Objectives}
\label{sec:orgf2df249}

Data science is about how to get data to work for us, to give us its
hidden treasures. Data science has been called "the sexiest job of
the 21st century". Even if you don't want to become a professional
data scientist, it’s helpful to master the basic concepts if you
want to succeed in today's highly data-driven business
environment. This courses focuses on: data science basics,
visualization and productivity tools. The course is for everyone who
is interested in becoming more data literate and growing their skill
stack. Besides short, synchronous lectures and practice sessions,
you can learn from weekly quizzes, exercises and a plethora of
other, custom-built online materials.

\subsection{Student Learning Outcomes}
\label{sec:orgebd37ce}

Students who complete DSC101 are able to:

\begin{itemize}
\item Organize data visually in a way that is clear and informative
\item Find and use data sets from the real world
\item Easily and quickly format data into graphs
\item Understand and present statistical information
\item Understand how modern productivity tools can help you
\item Complete an exploratory data analysis project in small steps
\end{itemize}

\subsection{Course requirements}
\label{sec:org6cc9362}

No prior knowledge required. Both the necessary programming and
statistical concepts are introduced in the course using examples and
simple mini-projects. Previous programming experience is useful but
not mission-critical. Curiosity is essential. You will gain data
literacy skills by taking this course.

\subsection{Grading system}
\label{sec:orge0bac37}

\begin{center}
\begin{tabular}{llll}
\hline
When & Description & Impact & Graded\\
\hline
Weekly & DataCamp assignment & 15\% & No\\
Weekly & Weekly tests & 15\% & No\\
Last week & Project presentation & 30\% & Yes\\
TBD & Final exam & 40\% & Yes\\
\hline
\end{tabular}
\end{center}

\subsubsection{Grading table}
\label{sec:orga3a96ea}

This table is used to convert completion rates into letter
grades. For the midterm results, letter grades still carry signs,
while for the term results, only straight letters are given (by
rounding up).

\begin{center}
\begin{tabular}{rll}
\hline
\textbf{\%} & \textbf{Midterm Grade} & \textbf{Final Grade}\\
\hline
100-98 & A+ & \\
97-96 & A & A\\
95-90 & A- & \\
\hline
89-86 & B+ & \\
85-80 & B & B\\
79-76 & B- & \\
\hline
75-70 & C+ & \\
69-66 & C & C\\
65-60 & C- & \\
\hline
59-56 & D+ & \\
55-50 & D & D\\
\hline
49-0 & F & F\\
\hline
\end{tabular}
\end{center}

\subsubsection{DataCamp assignments (15\%)}
\label{sec:orga031af6}

24 chapters of DataCamp lessons will be assigned to you. To
complete a chapter takes 20-30 minutes per week. If you complete
all chapters, you get five free data science certificates that you
can add to your resume (or to career management portals like
LinkedIn - see \href{https://www.linkedin.com/in/birkenkrahe}{my LinkedIn profile} for an example). Your DataCamp
assignment completion rate will enter the final grade cumulatively.

\subsubsection{Tests (15\%)}
\label{sec:org41fe1ec}

There will be 30 short multiple choice tests of 5 questions per
week. Your grade will be computed from your average completion rate
over all tests.

\subsubsection{Project presentation (30\%)}
\label{sec:org7bf628c}

In the last week, you present the results of an agile explorative
data analysis (EDA) team project. We use the agile Scrum
methodology throughout the term, which means that you will present
prototype results during four sprint reviews (about once every four
weeks), the last of which is the final product or project
presentation.

Note that only the final presentation will be graded according to
the grading table. Detailed grading criteria for the presentation
will be given in class in the form of a rubric.

\subsubsection{Exam (40\%)}
\label{sec:org55a3467}

The final exam will consist of a subset of the weekly test
questions, possibly with some slight variations to make it more
interesting. The basic idea is that you can use the tests to
prepare yourself for the exam. The completion rate of the final
exam will enter the final grade according to the grading table.

\subsection{Grading examples}
\label{sec:org67211f3}
\subsubsection{Example - Midterm grade}
\label{sec:org91a2fa4}

At midterms, student X has achieved the following results:

\begin{center}
\begin{tabular}{lll}
\textbf{Grade part} & \textbf{Weight} & \textbf{Result}\\
\hline
Tests & 15\% & 72\%\\
DataCamp assignment & 15\% & 100\%\\
\end{tabular}
\end{center}

Student X's midterm result is a "B+" (\texttt{86\%}).

\subsubsection{Example - Final grade}
\label{sec:org5da0e7a}

After the finals, student X has achieved the following results:

\begin{center}
\begin{tabular}{lll}
\textbf{Grade part} & \textbf{Weight} & \textbf{Result}\\
\hline
Tests & 15\% & 72\%\\
DataCamp assignment & 15\% & 100\%\\
Project presentation & 30\% & 95\%\\
Final exam & 40\% & 90\%\\
\end{tabular}
\end{center}

Student X's midterm result is an "A" (\texttt{90.3\%}).

\section{Standard Policies}
\label{sec:orgd224f32}
\subsection{Honor Code}
\label{sec:org4c3bd91}

All graded work in this class is to be pledged in accordance with
the Lyon College Honor Code. The use of a phone for any reason
during the course of an exam is considered an honor code
violation.

\subsection{Class Attendance Policy}
\label{sec:org67576e5}

Students are expected to attend all class periods for the courses
in which they are enrolled. They are responsible for conferring
with individual professors regarding any missed
assignments. Faculty members are to notify the Registrar when a
student misses the equivalent of one, two, three, and four weeks
of class periods in a single course. Under this policy, there is
no distinction between “excused” and “unexcused” absences, except
that a student may make up work missed during an excused
absence. A reminder of the college’s attendance policy will be
issued to the student at one week, a second reminder at two weeks,
a warning at three weeks, and notification of administrative
withdrawal and the assigning of an “F” grade at four
weeks. Students who are administratively withdrawn from more than
one course will be placed on probation or suspended.

\subsection{Disabilities}
\label{sec:org738bb00}

Students seeking reasonable accommodations based on documented
learning disabilities must contact Danell Hetrick in the Morrow
Academic Center at (870) 307-7021 or at danell.hetrick@lyon.edu.

\subsection{Harassment, Discrimination, and Sexual Misconduct}
\label{sec:orgb5e9e5d}

Title IX and Lyon’s policy prohibit harassment, discrimination and
sexual misconduct. Lyon encourages anyone experiencing harassment,
discrimination or sexual misconduct to talk to Lai-Monte Hunter,
Title IX Coordinator and Vice-President for Student Life, or
Sh’Nita Mitchell, Title IX Investigator and Associate Dean for
Residence Life, about what happened so they can get the support
they need and Lyon can respond appropriately.  Lyon is legally
obligated to respond to reports of sexual misconduct, and
therefore we cannot guarantee the confidentiality of a report,
unless made to a confidential resource (Chaplain, Counselor, or
Nurse). As a faculty member, I am required to report possible
Title IX violations and must provide our Title IX coordinator with
all relevant details.  I cannot, therefore, guarantee
confidentiality.

\subsection{College-Wide COVID-19 Policies for Fall, 2021}
\label{sec:org8f7e4ed}

Masks are mandated for all students in classrooms, laboratories and studios.  They remain optional for all persons on the Lyon campus in all other locations and outside.
Participation in community surveillance testing in mandatory.
Vaccines are STRONGLY encouraged for all faculty, staff, and students. Vaccines are NOT MANDATED for Lyon College community members.

Details specific to this course may be found in the subsequent pages of this syllabus. Those details will include at least the following:
A description of the course consistent with the Lyon College catalog.
A list of student learning outcomes for the course.
A summary of all course requirements.
An explanation of the grading system to be used in the course.
Any course-specific attendance policies that go beyond the College policy.
Details about what constitutes acceptable and unacceptable student collaboration on graded work.

\section{Course specific information}
\label{sec:org4bf840f}
\subsection{Assignments and Honor Code}
\label{sec:org3095d40}

There will be numerous assignments during the semester -
programming, lessons, tests, and sprint reviews. They are due at
the beginning of the class period on the due date. Once class
begins, the assigment will be considered one day late if it has
not been turned in.  Late programs will not be accepted without an
extension.  Extensions will not be granted for reasons such as:

\begin{itemize}
\item You could not get to a computer
\item You could not get a computer to do what you wanted it to do
\item The network was down
\item The printer was out of paper or toner
\item You erased your files, lost your homework, or misplaced your
flash drive
\item You had other coursework or family commitments that interfered
with your work in this course
\end{itemize}

Put “Pledged” and a note of any collaboration in the comments of
any program you turn in. Programming assignments are individual
efforts, but you may seek assistance from another student or me.
You may not copy someone else’s solution.  If you are having
trouble finishing an assignment, it is far better to do your own
work and receive a low score than to go through an honor trial and
suffer the penalties that may be involved.

What is cheating on an assignment? Here are a few examples:

\begin{itemize}
\item Having someone else write your program, in whole or in part
\item Copying a program someone else wrote, in whole or in part
\item Collaborating with someone else to the extent that your programs
are identifiably very similar, in whole or in part
\item Turning in a program with the wrong name on it
\end{itemize}

What is not cheating?  Here are some examples:

\begin{itemize}
\item Talking to someone in general terms about concepts involved in
an assignment
\item Asking someone for help with a specific error message or bug in
your program
\item Getting help with the specifics of language syntax
\item Utilizing information given to you by the instructor
\end{itemize}

Any assistance must be clearly explained in the comments at the
beginning of your program.  If you have any questions about this,
please ask or review the policies relating to the Honor Code.

Absences on Days of Exams:

Test “make-ups” will only be allowed if arrangements have been
made prior to the scheduled time.  If you are sick the day of the
test, please e-mail me or leave a message on my phone before the
scheduled time, and we can make arrangements when you return.

\subsection{Important Dates:}
\label{sec:org1a412a5}

\begin{center}
\begin{tabular}{ll}
\textbf{Date} & \textbf{Description}\\
\hline
August 30 & Last day to drop w/o record of a course\\
September 6 & Labor day (no classes)\\
October 2-5 & Fall break (no classes)\\
October 6 & Mid-semester grade reports due\\
October 13 & Last day to drop a course with a "W" grade\\
October 20 & Service day on campus (no classes)\\
Nobember 24-28 & Thanksgiving Break (no classes)\\
December 3 & Last day of class\\
December 6-10 & Final exams\\
December 15 & Final grades due\\
\end{tabular}
\end{center}

\subsection{Schedule and session content}
\label{sec:org95b3213}

\begin{center}
\begin{tabular}{lll}
\textbf{Date} & \textbf{Lectures} & \textbf{DataCamp}\\
\hline
17-Aug & \textbf{Course overview} & \href{https://learn.datacamp.com/courses/free-introduction-to-r}{Introduction to R:}\\
19-Aug & \textbf{Data science overview} & - Intro to basics\\
24-Aug & \textbf{The R shell} & - Vectors\\
26-Aug & \textbf{The R environment} & - Matrices\\
31-Aug & \textbf{Vectors} & - Factors\\
2-Sep & \textbf{Data frames} & - Data frames\\
7-Sep & \textbf{Factor vectors} & - Lists\\
9-Sep & \textbf{Lists in R} & \href{https://learn.datacamp.com/courses/intermediate-r}{Intermediate R}:\\
14-Sep & \textbf{\texttt{Nile} exploration} & - Conditionals\\
16-Sep & \textbf{Visualization} & - Loops\\
21-Sep & \textbf{Base R graphics} & - Functions\\
23-Sep & Writing functions & - The \texttt{apply} family\\
28-Sep & Iteration I & - Utilities\\
30-Sep & Fibonacci series & \\
7-Oct & \textbf{Literate Programming} & \href{https://learn.datacamp.com/courses/data-visualization-in-r}{Data Visualization:}\\
12-Oct & Conditions & - Base R graphics\\
14-Oct & EDA example I & - Different plot types\\
14-Oct & Linear regression I & - Adding details to plots\\
19-Oct & Object-orientation & - How much is too much?\\
21-Oct & EDA example II & - Plot customization\\
26-Oct & Packages & \\
28-Oct & Grammar of Graphics & \href{https://learn.datacamp.com/courses/introduction-to-regression-in-r}{Regression in R}:\\
2-Nov & Functional Programming & - Simple linear regression\\
4-Nov & Text mining I & - Predictions and models\\
9-Nov & Text mining II & - Assessing model fit\\
11-Nov & Linear regression II & - Simple logistic regression\\
16-Nov & Dates and times & \\
18-Nov & Coding style & \href{https://learn.datacamp.com/courses/text-mining-with-bag-of-words-in-r}{Text mining (Bag-of-Words)}\\
23-Nov & Logistic regression & - Basics of Bag-of-Words\\
30-Nov & Team presentations & - Word clouds and visuals\\
2-Dec & Team presentations & - Text mining techniques\\
2-Dec & Team presentations & - Text mining case study\\
\end{tabular}
\end{center}
\end{document}